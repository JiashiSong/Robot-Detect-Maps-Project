\documentclass{article}
\usepackage[utf8]{inputenc}
\begin{document}
\begin{center}    
{\huge Minutes of the Fourth Group Meeting \par}
\vspace{0.5cm}
{\large Group UG02 \par}
\vspace{0.5cm}
{\large Monday 7th August 2017 \par}
\vspace{0.5cm}
\end{center}

\begin{flushleft}
\begin{tabular}{ll}
{\bfseries Chair} & Andrew Graham \\
{\bfseries Minutes} & Brock Campbell \\
{\bfseries Members}
 & Alexander Good \\
 & Mitchell Mickan \\
 & Saxon Nelson-Milton \\
 & Jiashi (Josh) Song \\
{\bfseries Apologies} & Nathan Crowe\\
\end{tabular}
\end{flushleft}

\section{Time and Place}
The {\itshape fourth} meeting in {\itshape week 5} for the Software Engineering Group Project Group was held in the {\bfseries Atrium of Ingkarni Wardli}, at {\bfseries 2:04pm on Monday 21st August 2017.} 

\section{Issues and Actions}
	\subsection{Issues on Hold}
    Due to a pressing due date for the Software Requirements Specification draft document, many of the items in the Agenda have been postponed for another meeting either during the week or in the next meeting. These issues include:
    \begin{enumerate}
    \item Physical map design - how to physically design the map with respect to significant features.
    \end{enumerate}
	\subsection{Software Requirements Document}
    The main focus of the meeting was to get up to date on the SRS progress to ensure that it would be completed by the deliverable date (22nd August).
 	\begin{itemize}
    \item Alex claimed responsibility for answering questions related to the SRS format and section selection.
    \item It was decided that after the 10 minute SRS administrator review during the next client meeting, the remainder of the client meeting would be used to discuss the SRS with the client to see whether they agree with our requirements, and to share our plan for a prototype rover by the following week.
    \item Alex suggested Stimulus/response sections of the SRS should be translated to tabular form for readibility, and should be separated from the UI and other components.
    \item Mitchell noted that the performance requirements to be outlined in the SRS are to travel 500 metres and map the area within 15 minutes.
    \subsubsection*{Issue}
    Which turning should the rover use to turn such that all sensors have the least amount of chance of colliding with objects?
    \subsubsection*{Action}
    Saxon suggested a turning mechanism which focuses on the rover pivoting on the spot will be used.
    \subsubsection*{Issue}
    Mitchell put forward the problem of how will manual control work?
    \subsubsection*{Action}
    Brock stated a button press on a UI should be able to move the rover forward a set amount of distance, and the button will be polled for its state many times with each time moving the robot a bit more.   
    \end{itemize}
 
    \subsection{Development Sprint 1}
    Brock set up and suggested that we start developing basic movement functionality of the rover in the coming week. The scrum board for the first sprint was set up with the Product Manager being assigned to Brock, and the Scrum Master being assigned to Andrew. These roles will alternate for each sprint.

\subsection{Final Remarks}
It was decided that for the next week, the goals were to complete the SRS before the next client meeting, and to work through the first sprint to produce a prototype by the next group meeting.
     
\section{Adjournment}
	The next group meeting and will be held in the {\bf Atrium, Ingkarni Wardli} at {\bf 2:00pm on Monday 28th August 2017}.\newline
    The next client meeting and will be held in {\bf Room 462, Ingkarni Wardli} at {\bf 10:10am on Tuesday 22nd August 2017}.\newline
The meeting closed at 2:30pm.
\end{document}
