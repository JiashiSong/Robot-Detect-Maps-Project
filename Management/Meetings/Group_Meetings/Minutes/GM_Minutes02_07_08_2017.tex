\documentclass{article}
\usepackage[utf8]{inputenc}
\begin{document}
\begin{center}    
{\huge Minutes of the Second Group Meeting \par}
\vspace{0.5cm}
{\large Group UG02 \par}
\vspace{0.5cm}
{\large Monday 7th August 2017 \par}
\vspace{0.5cm}
\end{center}

\begin{flushleft}
\begin{tabular}{ll}
{\bfseries Chair} & Brock Campbell \\
{\bfseries Minutes} & Alexander Good \\
{\bfseries Members} & Nathan Crowe \\
 & Andrew Graham \\
 & Mitchell Mickan \\
 & Saxon Nelson-Milton \\
 & Jiashi (Josh) Song \\
{\bfseries Apologies} & None\\
\end{tabular}
\end{flushleft}

\section{Time and Place}
The {\itshape second} meeting in {\itshape week 3} for the Software Engineering Group Project Group was held in the {\bfseries Atrium of Ingkarni Wardli}, at {\bfseries 11:05am on Monday 31 July 2017.} 

\section{Issues and Actions}
	\subsection{Github Repository}
		\subsubsection*{Issue}
		Mitchell and Josh are still yet to be added into the group's Github repository. 
    	\subsubsection*{Action}
    	Email sent to Lectures, Amali and David.
     \subsection{Managers Role for Josh}
		\subsubsection*{Issue}
		As a new member of our group, Josh has now yet been formally allocated a managers role.
    	\subsubsection*{Action}
    	Assigned UI/Hardware Manager (subject to change)
\section{Management updates}
	\subsection{Documentation Updates}
    Brock discussed new centralized location of links to Overleaf documents on GitHub which are to be kept up to date. As elaborated by Andrew, the method for maintaining documentation on Github is now as follow:
   \begin{enumerate}
   	\item Create a new branch (called documentation, agendadocs or similar).
   	\item Add, commit and push the new/updated docs to the new branch.
   	\item Submit a pull request to merge the docs branch into master.
   	\item Approve and merge your own pull request to put the docs into master.
   	\item Finally, delete the new branch (after you merge it prompts you to do so).
   \end{enumerate}
   
	\subsection{Timesheets}
	Brock discussed the new Google Sheets file which has been created and shared with everyone to track contributions and elaborated on the fact that this is merely to make sure that everyone is doing something each week. This contributions are per week.

    \subsection{Scrum board}
Brock discussed using scrum boards for managing jobs, minutes, etc and the use of user stories in order to track requirements progress with 4 stages, To-do, progress, Quality Assurance, Complete. This will be does using issues on Github.\newline 

Andrew also gave a brief explanation of how action jobs on Github - Grab issue from project page, create branch, make changes, commit, make pull request, have approved and then merged into master.\newline 

Nathan queried about how to deal with incompatible merges? Andrew as Quality Assurance Manager will be dealing with branch conflicts but has stressed that everyone needs to be sure that they are always working on the most current master possible and {\em NEVER} commit to master.\newline

Andrew also gave a demonstration of Networks in Github.	 

\section{Client Meeting Preparation}
\subsection{Questions for Client}
Group discussed questions posed in agenda for tomorrows meeting, prioritized each of them and organized them into categories.

Mitchell queried whether whole area must be mapped and what overrides may need to be implemented when asked to leave boundary.

Also discussed was accuracy of mapping, how to maintain the rovers location without the use of GPS and what to do in case of loss of control of rover.

\section{Weekly Goals}
Brock has set out the following goals for the group this week.
  \begin{itemize}
\item Documentation Manager
  \newline Task: Organize OverLeaf documents into projects by their type. So far, the 3 main documents (SPMP, SRS, SDD) are all in their own project (good!). It would be good to have Client meeting agendas and minutes in a separate project to Group meeting agendas and minutes. (prority: low - not necessary).
  \item Testing Manager
  \newline Task: Install and get familiar with the JUnit testing suite. Want to be proficient enough to help others in the group set it up and use it. (priority: low - not coding yet).
  \item Requirements Manager
  \newline Task: Draw up a requirements dependency tree, grouping similar requirements. Goal is to separate requirements into blocks that will produce a working component of the system, so we can focus on one component at a time. (start: 08/08/2017 after client meeting. priority: high - required ASAP).
  \item Design Manager
  \newline Task: From the vague description of the requirements, converse with the Requirements Manager and begin thinking about the overall design of the system. Don't need a UML diagram outlining each component of the system, just a general idea of what we will need to implement on a low level. (priority: medium)
  \item Quality Assurance Manager
  \newline Task: Get familiar with Git workflows and the commands required to follow them to be able to guide others. Perhaps draw up a small document (based on Atlassian tutorials) that outlines basic git commands to use? (priority: low - not coding yet).
  \item Project Manager
  \newline Task: Start writing the Software Project Management Plan. (priority:low). Work on the client meeting agenda for 8/8/2017. (priority: extreme).
  \item Everyone
  \newline Task: Install LeJos (Eclipse or IntelliJ plugin). Get familiar with the syntax, read a bit of the docs  and perhaps some code online to figure out what we can and can't do. (prioirty: low - the more you read the better off you will be later).
  \newline Task: Smash out the Software Requirements Specification document. (Due 2 weeks from now - unconfirmed?)
  \end{itemize}

\section{Adjournment}
	The next group meeting and will be held in the {\bf Atrium, Ingkarni Wardli} at {\bf 2:00pm on Monday 14th August 2017}.\newline
    The next client meeting and will be held in {\bf Room 462, Ingkarni Wardli} at {\bf 10:10am on Tuesday 8th August 2017}.\newline
The meeting closed at 12:10pm.
\end{document}
