  %%%%%% Weekly Meeting Agenda 08/08/2017
  \documentclass[11pt, a4paper]{article}
  \usepackage{times}
  \usepackage{ifthen}
  \usepackage{amsmath}
  \usepackage{amssymb}
  \usepackage{graphicx}
  \usepackage{setspace}

  %%% page parameters
  \oddsidemargin -0.5 cm
  \evensidemargin -0.5 cm
  \textwidth 15 cm
  \topmargin -1.2 cm
  \textheight 22 cm

  \renewcommand{\baselinestretch}{1.4}\normalsize
  \setlength{\parskip}{0pt}

  %% Meeting Details
  \newcommand{\meetingno}{ Second }
  \newcommand{\meetinglocation}{ Atrium, Ingkarni Wardli } %% Location
  \newcommand{\meetingdatetime}{ 11:00am 07/08/2017 } %% Date and time
  \newcommand{\meetingchair}{ Brock } %% Chair
  \newcommand{\meetingminutes}{ Alex } 

  \begin{document}

  %%%mention the no, time, and venue of the meeting
  \noindent The {\em \meetingno} Software Engineering Group Project weekly meeting will be held in {\bf \meetinglocation } at {\bf \meetingdatetime }.

  \vspace*{15pt}

  \begin{center}
  \huge \bf Agenda
  \end{center}

  \begin{flushleft}
  
  %%%first, nominate a chair for the meeting. We suggest that each member at least has one chance as the chair.
  \section*{Chair: \meetingchair }
  \section*{Minutes: \meetingminutes }
  

  \vspace*{10pt}

  %%%if some students cannot make the meeting due to some reasons, their names should appear here.
  \section{Apologies}

  %%%short presentation about the work of previous week or any milestone specified in the course.
  \section{Presentation}
  N/A
  
  %%%any schedules for this meeting should go next, each with a separate section.
  %%%for example, the first meeting is about requirement elicitation, like the following.
  \section{Management updates}
  \subsection{Documentation Updates}
  \begin{itemize}
  \item Documentation links will now be placed in a centralized location on GitHub. These links need to be kept up to date.
  \item Furthermore, submitting any new documentation or documentation updates will follow the defined format:
  \begin{enumerate}

  \item Create a new branch (called documentation, agendadocs or similar).
  \item Add, commit and push the new/updated docs to the new branch.
  \item Submit a pull request to merge the docs branch into master.
  \item Approve and merge your own pull request to put the docs into master.
  \item Finally, delete the new branch (after you merge it prompts you to do so).
  
  \end{enumerate}
  
  \end{itemize}
  \subsection{Timesheets}
  \begin{itemize}
  \item A Google Sheets file has been created and shared with everyone to track our contributions. You can find the timesheet link in the README file.
  \item Each week, please add your contributions (tasks you complete, issues/ features you address), the outcome of those contributions and and estimated time spent on each task.
  \item Please fill in any contributions you made last week.
  \item Note you can count partial contributions (e.g. added requirements section 2.4,2.5).
  \item This format is open for debate. Do not be worried about having less hours than someone (or less tasks than someone), it's just a way to make sure that everyone is doing at least something each week.
  \end{itemize}
  
  \subsection{Scrum board}
  On the GitHub repo Projects page, there will be a couple of scrum boards to track feature and documentation progress.
  
  \subsection{Repository updates}
  We have a predefined format for how we will operate using GitHub. Andrew will elaborate on this.

  %%%if there are more subissues, make them as subsections.
  \subsection{Discuss requirements}
  We need to discuss the requirements of the project, and what questions we are going to ask on the first client meeting on 08/08/2017.

  %%%more issues should make it like the above one.
  \section{Other Issues}
  \subsection{Manager weekly tasks}
  \begin{itemize}
  \item Documentation Manager
  \linebreak Task: Organize OverLeaf documents into projects by their type. So far, the 3 main documents (SPMP, SRS, SDD) are all in their own project (good!). It would be good to have Client meeting agendas and minutes in a separate project to Group meeting agendas and minutes. (prority: low - not necessary).
  \item Testing Manager
  \linebreak Task: Install and get familiar with the JUnit testing suite. Want to be proficient enough to help others in the group set it up and use it. (priority: low - not coding yet).
  \item Requirements Manager
  \linebreak Task: Draw up a requirements dependency tree, grouping similar requirements. Goal is to separate requirements into blocks that will produce a working component of the system, so we can focus on one component at a time. (start: 08/08/2017 after client meeting. priority: high - required ASAP).
  \item Design Manager
  \linebreak Task: From the vague description of the requirements, converse with the Requirements Manager and begin thinking about the overall design of the system. Don't need a UML diagram outlining each component of the system, just a general idea of what we will need to implement on a low level. (priority: medium)
  \item Quality Assurance Manager
  \linebreak Task: Get familiar with Git workflows and the commands required to follow them to be able to guide others. Perhaps draw up a small document (based on Atlassian tutorials) that outlines basic git commands to use? (priority: low - not coding yet).
  \item Project Manager
  \linebreak Task: Start writing the Software Project Management Plan. (priority:low). Work on the client meeting agenda for 8/8/2017. (priority: extreme).
  \item Everyone
  \linebreak Task: Install LeJos (Eclipse or IntelliJ plugin). Get familiar with the syntax, read a bit of the docs  and perhaps some code online to figure out what we can and can't do. (prioirty: low - the more you read the better off you will be later).
  \linebreak Task: Smash out the Software Requirements Specification document. (Due 2 weeks from now - unconfirmed?)
  \end{itemize}
%%%finally, specifies time of next meeting
  \vspace*{10pt}
  \noindent Note: Next meeting to be held on 13 August 2017.
  \noindent\linebreak Note: Meeting start-time 11.05am. End time 12.10pm.
  \end{flushleft}

  \end{document}
