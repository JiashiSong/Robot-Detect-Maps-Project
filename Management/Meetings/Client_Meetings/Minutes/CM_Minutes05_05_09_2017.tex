\documentclass{article}\usepackage[utf8]{inputenc}\begin{document}\begin{center}    {\huge Minutes of the 5th Client Meeting \par}\vspace{0.5cm}{\large Group UG02 \par}\vspace{0.5cm}{\large Tuesday 12th September 2017 \par}\vspace{0.5cm}\end{center}
\begin{flushleft}\begin{tabular}{ll}{\bfseries Chair} & Nathan Crowe \\{\bfseries Minutes} & Jiashi Song \\{\bfseries Members} & Andrew Graham  \\ & Alexander Good \\ & Mitchell Mickan \\ & Saxon Nelson-Milton \\ & Brock Campbell \\{\bfseries Apologies} & None\\\end{tabular}\end{flushleft}
\section{Time and Place}The {\itshape 5th} client meeting in {\itshape week 8} for the Software Engineering Group Project Group was held in the {\bfseries Ingkarnli Wardli 462}, at {\bfseries 10:10am on Tuesday 12th September 2017.} 
\section{Quorum Announcement} After a brief introduction by the client, it was announced that the first part of the meeting was to be dedicated to the presentation of the first milestone, and that any discussion can proceed afterwards.     \section{Summary of Previous Meeting}  Not Applicable - This was not discussed.
\section{Milestone 1 Presentation}Milestone 1 was presented, starting with the sensor functionality. It was shown the sensors were outputting data, and that this data could be shown as valid through reading it from the rover prototype LCD screen. The client was happy that the sensors were functional, but suggested that he didn't require the LCD functionality beyond testing. 
Manual rover control was then demonstrated through toggling of the GUI manual button. Several movements were demonstrated. The client was happy with the progress in this regard. 
To follow, the physical demonstration map was shown. The client suggested that the landing trail should be dragging rather than a few strokes. The client also suggested that the radiation did not seem that realistic. It was encouraged that the radiation should take the shape of an ellipse, to show that solar winds encourage deviation of the radiated materials away from the crash site. To end, the landing site was discussed. The client reported that they are not concern as to where we start, as long as we can demonstrate the required capabilies. 
\section{Milestone 1/2 forms}The client was asked to get the first milestone form back soon, so that it can be signed. It was suggested that this would avaible in the next meeting. 
The second milestone form was then presented. The client suggested that we remove talk on sensor data, as he doesn't particularly care that is or what it means. The client wanted more specifics on what is meant by avoid obstacles. The client then asked what needs to be done after milestone 2 is finished. Overall the client was happy with this milestone form. 
\section{DTD Questions}Some Questions regarding the DTD were asked.
To start, it was asked if the "position of the rover" referred to the apollo rover or the prototype rover. The response was that it refers to the protoype rover, but during demonstration, the development team can choose where to start. 
Then it was asked how points outside the map boundary are dealt with 
\section{SDD}It was discussed as to whether the SDD and milestone 2 goals would be presented together. It was confirmed that they would be presented together. 
\section{Next Meeting Deliverables}The second milestone will need to presented, along with the SDD. 
\section{Adjournment}    The next client meeting and will be held in {\bf Room 462, Ingkarni Wardli} at {\bf 10:10am on Tuesday fifth of September 2017}.The meeting closed at 10:38am.



\end{document}
