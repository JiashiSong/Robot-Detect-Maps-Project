\documentclass{article}
\usepackage[utf8]{inputenc}
\begin{document}
\begin{center}    
{\huge Minutes of the Second Client Meeting \par}
\vspace{0.5cm}
{\large Group UG02 \par}
\vspace{0.5cm}
{\large Tuesday 15th August 2017 \par}
\vspace{0.5cm}
\end{center}

\begin{flushleft}
\begin{tabular}{ll}
{\bfseries Chair} & Alexander Good \\
{\bfseries Minutes} & Andrew Graham \\
{\bfseries Members} & Nathan Crowe \\
 & Brock Campbell \\
 & Mitchell Mickan \\
 & Saxon Nelson-Milton \\
 & Jiashi (Josh) Song \\
{\bfseries Apologies} & None\\
\end{tabular}
\end{flushleft}

\section{Time and Place}
The {\itshape second} client meeting in {\itshape week 4} for the Software Engineering Group Project Group was held in the {\bfseries Ingkarnli Wardli 462}, at {\bfseries 10:10am on Tuesday 15th August 2017.} 

\section{Quorum Announcement}
	After a brief introduction by the client, it was announced that the first 15 minutes would be dedicated to identifying the group management strategies, and that the meeting could commence.
    
\section{Summary of Previous Meeting}
 	Not Applicable - This was not discussed.


\section{Personal Introductions \& Responsibilities}
	\textbf{Brock Campbell: Project Manager }
	\begin{itemize}
	\item Manages the project schedule, performs risk assessment and coordination. Ensures that everyone is doing their job, and being the go-to person if someone is unable to do their job. 
	\end{itemize}
	\textbf{Nathan Crowe: Test Manager}
	\begin{itemize}
    \item Oversees unit testing of all code including the creation of testing scripts. Ensures a testing standard (JUnit) is followed. It was noted that both physical and software testing would need to occur.
	\end{itemize}
	\textbf{Andrew Graham: Quality Assurance Manager}
	\begin{itemize}
    \item Manages code and documentation quality through issue tracking and pull requests on GitHub. Ensures that coding standards are followed. It was recommended that a flexible stance be taken towards coding style that reflected the group's personal styles.
	\end{itemize}
	\textbf{Mitchell Mickan: Requirements Manager}
	\begin{itemize}
	\item Has the final say on all requirements and manages the addition/ removal and restructuring of new requirements. It was noted that this position may become easier as the semester progresses due to the clarification of requirements.
	\end{itemize}
    \textbf{Alexander Good: Documentation Manager}
    \begin{itemize}
    \item Manages, organizes and proof reads all documentation. Ensures that documentation standards are followed. It was noted that not all documentation is actually done by the manager, but rather he is to oversee the final documentation that is added to github.
    \end{itemize}
    \textbf{Saxon Milton: Design Manager}
    \begin{itemize}
    \item Manages all aspects related to the design of the system. Has the final say for how to manage components in the system and how they will interact with each other.
    \end{itemize}
    \textbf{Josh Song: Hardware/UI Manager}
    \begin{itemize}
    \item Designs and implements hardware components, redesigning if needed. Manages the format of the UI component of the system.
    \end{itemize}


\section{Requirements Elicitation}
	Requirements elicitation was grouped into 5 distinct categories - rover, mapping, mission, design and uncategorized requirements. 
    
    \subsection{Lunar Rover Requirements}
    	\textbf{Q:} Given that we only plan to utilize 3 wheels on our rover (including the ball bearing), how should we interpret the meaning of "unable to recover"? \newline
        \textbf{A:} Unable to recover in this scenario would be when 2 or more wheels cross a line that delineates a crater. Using the table as a practical example, it can be seen that when 2 or more wheels of the rover enter a crater, the rover would be unable to recover itself.\newline
    
    	\textbf{Q:} How should manual control be handled? The options that we could think of include physically moving the prototype rover, having an autonomous return to base option, or including manual controls on the UI interface. \newline
        \textbf{A:} Optimally, we would like to have manual controls on the UI interface to remotely control the rover, as this best mimics the actual scenario. Alternatively, a button on the actual prototype rover could be used as a fallback. \newline   
        
        \textbf{Q:} Given our limited budget, what materials will we have available to us to create a prototype map?  \newline
        \textbf{A:} We are to fund our own materials, and can get an A1 sheet and markers, or create an  A1 sheet by sticking A4 sheets together. We may use any of the cardboard or sheets present in the lab if they are available. Our prototype map needs to be explained clearly, so that the final assessment map can use similar protocols for delineating features. \newline  
\section{Physical Design}
	Alex quickly explained the design of the physical model.        
         
\section{Adjournment}
	The next group meeting and will be held in the {\bf Atrium, Ingkarni Wardli} at {\bf 2:00pm on Monday 21st August 2017}.\newline
    The next client meeting and will be held in {\bf Room 462, Ingkarni Wardli} at {\bf 10:10am on Tuesday 22th August 2017}.\newline
The meeting closed at 10:35pm.
\end{document}
