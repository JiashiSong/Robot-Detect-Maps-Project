\documentclass[11pt, a4paper]{article}
\usepackage{times}
\usepackage{ifthen}
\usepackage{amsmath}
\usepackage{amssymb}
\usepackage{graphicx}
\usepackage{setspace}
\usepackage{fancyhdr}

%%% page parameters
\oddsidemargin -0.5 cm
\evensidemargin -0.5 cm
\textwidth 17 cm
\topmargin -2 cm
\textheight 25 cm

\renewcommand{\baselinestretch}{1.1}\normalsize
\setlength{\parskip}{0pt}

%% Meeting Details

\pagestyle{fancy}
\lhead{Mitchell Mickan SEP UG02}
\rhead{a1669783}


\begin{document}
	
	\begin{center}
		\huge \bf Software Requirements Specification Review
	\end{center}
	
	\paragraph{} {All the requirements in the document follow a predefined template throughout the document which helped us to keep things consistent. This template includes clear verification criteria that are used to confirm that the requirement was properly completed. Implementation priority was one of the areas of requirements that may have been less consistent as we did not have a set definition of what each priority level meant.  Cross references between majority of the requirements are correct however there are some inconsistencies with some requirements that have no dependencies having nothing written rather than specifying none. All requirements to made sure to only have one requirement and any that had multiple were split for simplicity.}
	\paragraph{} {One major area that could’ve been improved was the introduction to the document. A clearer explanation of what each section consisted of would have been more helpful to a reader that is unfamiliar with the document rather than relying on the section’s name to describe itself. This would be especially useful for developers who had no part in creating the document but needed more information quickly about what they were developing. Minor improvements could also be added to the functional requirements as the way we wrote them was to have all the stimulus/ response pairs together before going into detail about the actual requirements. Because of this it wasn’t as clear which pair belonged to each requirement. This could easily be fixed by having an identifier for the corresponding requirement next to each pair. Overall the layout of the document was done well with everything in a clear logical order with only certain sections that could do with some expansion.}
	\paragraph{} {While writing our SRS we had a system that meant above each section we would have a comment telling us the author of that section and any reviewers. This greatly helped the reviewing process as we knew each area that had or hadn’t been reviewed and as such we could easily verify that each section had been read over by another person (ideally multiple). Because of this our writing was free of any grammatical errors and style and conciseness was consistent despite having seven people contributing to the document. }
	\paragraph{} {One area that could be improved was our grouping of requirements. In our functional requirements we split these into different categories however we found that some of these categories had overlapping requirements. Our solution was to simply not bother rewriting a requirement that was already in another category however upon reflection we should have chosen better categories to split these requirements so that overlapping did not occur. Besides this issue there is only one case where a requirement would have been a better fit in a different category which would be moving the minimum distance requirement from user requirements to performance requirements. }
	\paragraph{} {Safety requirements are an area where we deviated from the requirement template and instead wrote a paragraph about the safety of this project. In hindsight I don’t think this was a good idea and this paragraph could’ve been summarized into a requirement in the same template as others which would have been better for consistency. The rest of our quality requirements cover all the necessary areas, going into good level of detail about quality software implementation, security details, and reusability of project after completion.}
	\paragraph{} {All of the main requirements to do with the rover have been covered thoroughly except our system requirements. We lacked a detailed section specifying things like the operating system needed, actual components of the rover, and memory etc. This was likely overlooked due to our assumption that our client knew the hardware we were working with which in retrospection was not a good assumption. }
	
	
	\newpage
	\begin{center}
	\huge \bf Presentation Self Evaluation
	\end{center}
	\paragraph{} {One thing I believe I did well was that I spoke clearly and with clarity. Throughout the presentation I kept a consistent talking speed and made sure to articulate my words clearly so that I was easy to understand. I believe this was also helped by the fact that this was a recorded video and not a live one so I was unaffected by nerves. }
	
	\paragraph{} {If I did this presentation again I think I would get a proper script ready beforehand and then convert this to main talking points. This would help me have a better idea of what I want to say when I got to each talking point as I would have the main script to reference. In my case I only had dot points to go off of and this resulted in me retaking the video multiple times. Another thing I would do is practice more as although I did practice multiple times you can never prepare enough.}
	
	\paragraph{} {In terms of content of the presentation I think covered the main points of my review fairly well however in certain areas I think I could have gone into more detail. For example when talking about the inconsistencies in the dependencies I could have talked about how we ordered our requirements well to make the dependencies flow well throughout the document. I also think I could have covered more on the introdution area of the document about what we did well and what we did poorly. }
	
	\paragraph{} {I believe that my eye contact was mostly very good throughout the video. I attribute this to having what I wanted to talk about in dot points rather than script form as this meant I didn't have to keep glancing away too often. The only section my eye contact was poor was at the start when I had a list of categories to directly read out. In hindsight this could have easily been memorized for better eye contact.}
	
	\paragraph{} {Throughout the video my guestures were very poor and barely used at all. I think this is mostly due to the fact that due to my setup it was easiest to take the video while sitting down which led me to not make any guestures as it feels more unnatural. If I was standing up I believe my guestures would have been more visible and would've helped to communicate my talking points more. If I did this presentation again while sitting down (Ideally I would be standing) I would definitely make sure that there is plenty of space around me and focus more on expressing myself accompanied by guestures. }
	
	\paragraph{} {Before starting my practice I made sure I had key talking points to go off. From here I started my presentation adlib using these talking points with recording turned off to get a good idea of what I wanted to say. Once I had a good idea of what I wanted to say I did several full takes to get used to doing my review. Finally I did two recordings and chose the best one. Overall I believe this was a sufficient level of practice which definitely helped me in the final recordings. Despite this I still feel like more practice would have helped me stumble over words less and memorize the script more so that I could have had better eye contact while doing the presentation. In the future I would likely have more key points to go off of and then cut these downs as I better understand what I want to say. }
	\paragraph{} {Overall I am pleased with my presentation and believe I did a good job getting my main talking points across. If I did this presentation again I would definitely do it standing so I can focus more on my guestures. I would also make sure to have more detailed talking points as it is easier to cut down the amount of content rather than add things in at the end. }
	\\
	\\
	Youtube Link: https://www.youtube.com/watch?v=uYvm5AkAe0o
	
	
	
	
\end{document}