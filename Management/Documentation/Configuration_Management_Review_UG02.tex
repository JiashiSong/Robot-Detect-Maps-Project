\documentclass[11pt, a4paper]{article}
\usepackage{times}
\usepackage{ifthen}
\usepackage{amsmath}
\usepackage{amssymb}
\usepackage{graphicx}
\usepackage{setspace}
\usepackage{fancyhdr}

%%% page parameters
\oddsidemargin -0.5 cm
\evensidemargin -0.5 cm
\textwidth 17 cm
\topmargin -2 cm
\textheight 25 cm

\renewcommand{\baselinestretch}{1.1}\normalsize
\setlength{\parskip}{0pt}

%% Meeting Details

\pagestyle{fancy}
\lhead{Andrew Graham SEP UG02}
\rhead{a1687414}


\begin{document}
	
	\begin{center}
		\huge \bf Configuration Management Plan Review
	\end{center}
	
	\paragraph{} {This document reviews the configuration management strategies and surrounding plans in place for the UG02 Software Engineering Project. Items included in this review include the management roles assigned to group members, the management guidelines in place for the group and also the individual guides that have been created for use.}
	\paragraph{} {A major strength of the configuration management plan is the delegation of a dedicated quality assurance (QA) manager. The expectations for this manager are outlined in the SRS, along with the expectations for other group members who contribute to the final product. As a result of this, several guides have been created for this group project. These include a unique style guide for the group which outline coding style recommendations for the group. The other guide created is a github tutorial, which specifically outlines how to use github in a feature driven workflow. Both these guides provide the basis for developing a quality product that is able to be understood by new developers and easily extended due to its consistent style. }
	\paragraph{} {While there is a dedicated QA manager for the group, there are also processes in place to ensure that all group members are actively involved in quality assurance. The group has been briefed on how to perform peer code reviews on pull requests, and how to use software version software correctly, and this has been documented in the group SPMP. This will result in a constant and thorough review of all production code, and ensure that code quality, extensibility and structure follows the recommended guidelines. A potential point for improvement would be to schedule fortnighly or monthly code audits to assess the entire code base as a whole. This would further ensure a consistent style and code quality across the entire repository. }
	\paragraph{} {An area for improvement in the configuration management plan is how software releases will be handled by the group. This point seems to have been overlooked in our management plan, with only a brief mention of software releases being made in . It would be good to have a process in place to identify when a software release needs to be made (ie at the end of every sprint or milestone) and how to create and name a new release.  }
	\paragraph{} {Documentation release times and protocols have been well documented with a writing and review process in place for all documents which are to be released. This includes the use of templates to ensure consistent styling throughout our documentation, and a commenting system that allow for multiple writers and reviewers of the final documentation. The interaction between our documentation platform (Overleaf) and github have been defined, with guidelines regarding how to collaborate, name and upload to github being provided.  }
	\paragraph{} {Another area which could be improved further is our change management action plan. While potential changes and their impact have been analyzed, currently there is no grouping which collates these potential changes into a single section. Instead, these items are loosely scattered throughout our risk management plan, and together they form a rough process for identifying and managing potential changes. The change control process for our group is loosely defined, as it is expected that each potential change could require a different approach depending on its nature and severity. }
	\paragraph{} {Protocols for using version control software are well defined. A guide for the group has been created which describes the usage of github with feature driven development and also utilizes an example development cycle for better understanding. This guide also outlines the naming protocol for branches and releases within github. }	
	
	\newpage
	\begin{center}
	\huge \bf Presentation Self Evaluation
	\end{center}
	\paragraph{} {In this presentation, I believe I was able to articulate myself clearly and cover the main points of our configuration management plan in a concise manner. I took advantage of the fact that we were allowed to have multiple takes of our video to polish my presentation style and practice it until I was happy with the fluency of my speech. A set of refined notes also ensured that I had cues to follow throughout the presentation if needed. Multiple takes definitely allowed me to settle my nerves in a way which would not be possible in a live video.}
	
	\paragraph{} {Looking at my presentation, eye contact was maintained throughout the majority of the presentation. In the beginning, it felt strange to be focussing on a camera as I talked, and my eyes tended to wander, but as I practiced more I was able to maintain my focus on the camera. I also attribute my eye contact to the fact that I was not referring to a written script, but rather talking off the top of my head as the ideas came. I did have a set of notes in case I was to falter during the presentation, and I did look at these briefly towards the end of the video as I was presenting a section that I had practiced less. Overall, I think my eye contact was natural and consistent throughout the presentation.}
	
	\paragraph{} {During earlier takes, I had struggled to talk for the 5 minutes required for the presentation. However, as I became more confident with each take, I was able to flesh out what I was talking about and explain some details that I had previously missed. Being able to review my previous takes was extremely beneficial, as it is not something I frequently get to do. Listening and watching myself, I was able to identify areas where I struggled to articulate myself and focus on these areas to improve them. I also think I was talking too fast in earlier takes, so I made a concerted effort to slow my final presentation to allow myself to articulate myself more clearly. }
	
	\paragraph{} {Throughout the presentation, my tone was varied which helped to keep the talk engaging for the listener. In earlier takes, I found that I was a little monotonous during parts that I had already practiced a number of times, so I made an extra effort to be engaging during those sections. I did have some hand gestures which are not seen on camera, and I think if I was to re-take the video, it would be good to include my upper body in the shot so body language and hand gestures could be captured in the video.}
	
	\paragraph{} {The content of the video was clearly presented and I covered all the main points that I had listed during the presentation. In the beginning, I presented confidently and knew what I wanted to say and how I would say it. Towards the end of the video, I think my presenting became a little less fluent, as I had practiced this part less and was a little unsure of how I wanted to present my points. With further practice, some of these little inconsistencies could have been ironed out. }
	
	\paragraph{} {Overall, I am happy with the quality of my presentation. With even more practice, it could have become more polished, but I had reached a point where my performance was more than satisfactory. The major points of our configuration plan were presented accurately and succinctly, with natural speech and body language to accompany it. Points of improvement could include standing back from the camera so more body language and hand gestures could be captured, and practicing the tail end of my speech more to iron out any faltering towards the end.}
	\\
	\\
	Youtube Link: https://youtu.be/gg2XlIBpi3Q
	
	
	
	
\end{document}