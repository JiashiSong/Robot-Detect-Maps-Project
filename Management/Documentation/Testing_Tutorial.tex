\documentclass[11pt, a4paper]{article}

\usepackage{listings}
\usepackage{color}
\usepackage{hyperref}

\begin{document}

\definecolor{dkgreen}{rgb}{0,0.6,0}
\definecolor{gray}{rgb}{0.5,0.5,0.5}
\definecolor{mauve}{rgb}{0.58,0,0.82}

\lstset{frame=tb,
	language=Java,
	aboveskip=3mm,
	belowskip=3mm,
	showstringspaces=false,
	columns=flexible,
	basicstyle={\small\ttfamily},
	numbers=none,
	numberstyle=\tiny\color{gray},
	keywordstyle=\color{blue},
	commentstyle=\color{dkgreen},
	stringstyle=\color{mauve},
	breaklines=true,
	breakatwhitespace=true,
	tabsize=4
	}

\begin{center}    
{\huge Testing Tutorial \par}
\vspace{0.5cm}
{\large By Nathan Crowe for Group UG02 \par}
\vspace{0.5cm}
{\large August 27, 2017 \par}
\vspace{0.5cm}
\end{center}

\section{Purpose}
	This is a tutorial to create testing scripts for the Lunar Rover, with the aid of JUnit and Maven.
\section{Intended Audience}
	UG02, and staff related to the Software Engineering \& Project course at University of Adelaide.
\section{Scope}
	This document contains basic methods to create testing scripts which thoroughly test the Lunar Rover, with the aid of JUnit and Maven.
\section{Adding a New Test File}
	Add 'newTestFile.java' (or any other name *.java) to \textbackslash src\textbackslash test\textbackslash java of the related project (such as Rover, or UI). \\
	Also ensure the following dependency is present in pom.xml:
	\lstset{language=XML}
	\begin{lstlisting}
<dependencies>
  <dependency>
    <groupId>junit</groupId>
    <artifactId>junit</artifactId>
    <version>4.12</version>
    <scope>test</scope>
  </dependency>
</dependencies>
	\end{lstlisting}
	
\section{Basic Test}

	\lstset{language=Java}
	\begin{lstlisting}
import org.junit.Test;
import static org.junit.Assert.assertTrue; // see http://junit.org/junit4/javadoc/latest/

public class newTestFile {
	
	// degenerate true boolean test
	// tests that true is true...
	@Test
	public void testTrue() {
		assertTrue( true );
	}
}
	\end{lstlisting}
	The above very basic test tests that true is true.
	\begin{lstlisting}
import org.junit.Test;
	\end{lstlisting}
	The above line will be present for all JUnit test scripts. It is related to '@Test'.\\
	'@Test' indicates the upcoming method acts as a test.\\
	'@BeforeClass' indicates the upcoming method should run once before all tests.\\
	'@Before' indicates the upcoming method should run before each test.\\
	'@After' indicates the upcoming method should run after each test.\\
	'@AfterClass' indicates the upcoming method should run once after all tests.\\
	'@Ignore' indicates the upcoming method should be ignored (but compile nonetheless).\\
	'@BeforeClass' requires the import:
	\begin{lstlisting}
import org.junit.BeforeClass;
	\end{lstlisting}
	The other four tools require the import you would imagine.
	
	\begin{lstlisting}
import static org.junit.Assert.assertTrue;
	\end{lstlisting}
The above line is related to the assertTrue(Boolean value) function. Other assert functions exist, such as pertaining to equality between two parameters. They can be browsed from \url{http://junit.org/junit4/javadoc/latest/}.

\section{Exception Test}
	\begin{lstlisting}
import org.junit.Test;

public class newTestFile {
	// tests the existence of the Rover class
	@Test(expected = Rover.class)
	public void testRover(){
		assertTrue( true );
	}
}
	\end{lstlisting}

\section{Time Test}
	\begin{lstlisting}
import org.junit.Test;

@Test(timeout=3000) // 3 seconds to pass test
public void testMovement() {
	// wait until rover has moved 5mm
	// assumes that the rover is facing a wall
	Float initialPos = rover.getUsensor().getDistance();
	while( Math.abs(initialPos - rover.getUsensor().getDistance()) < 5){}
}
	\end{lstlisting}
	
\section{Notes}
	The design of the rover software may need to be expanded (hopefully orthogonally) to accommodate testing in software, as the current design has a lack of accessor methods.\\
	Testing in hardware may require writing a 'test runner' (normally required in general for JUnit, but Maven makes this unnecessary) to replace or expand upon the 'logic' class in a special hardware testing mode.

\end{document}
